% Code borrowed from my compilers colleagues for building a draft version with comments in the margin
\def\paperversiondraft{draft} \def\paperversionblind{normal}
\def\paperversioncameraIEEE{cameraIEEE}

% % If no draft paper-version is requested, compile in 'normal' mode
\ifx\paperversion\paperversiondraft
   \def\ClassReview{review,}
\else
  \ifx\paperversion\paperversioncameraIEEE
  \else
    \def\paperversion{normal}
  \fi
   \def\ClassReview{}
\fi

%%%%%%%%%%%%%%%%%%%%%%%%%%%%%%%%%%%%%%%%%
% JFP Specifics 
%%%%%%%%%%%%%%%%%%%%%%%%%%%%%%%%%%%%%%%%%
%%% save the original kernel definitions

\let\latexdocument\document
\let\latexenddocument\enddocument
\documentclass[\ClassReview acmsmall,screen,prologue,dvipsnames,style=authoryear,anonymous]{acmart}
%%% recover the original definitions
\let\document\latexdocument
\let\enddocument\latexenddocument

\bibliographystyle{ACM-Reference-Format}
\setcitestyle{authoryear}

%%%%%%%%%%%%%%%%%%%%%%%%%%%%%%%%%%%%%%%%%
% Agda tools
%%%%%%%%%%%%%%%%%%%%%%%%%%%%%%%%%%%%%%%%%
% wrap texttt lines
\sloppy

%%%%%%%%%%%%%%%%%%%%%%%%%%%%%%%%%%%%%%%%%%
% BEGIN draft version config
%%%%%%%%%%%%%%%%%%%%%%%%%%%%%%%%%%%%%%%%%%
\usepackage{colortbl}
\usepackage{xargs}
\usepackage{lipsum}
\usepackage[textsize=tiny]{todonotes}
\usepackage{xparse}
\usepackage{xifthen, xstring}
\usepackage[normalem]{ulem}
\usepackage{xspace}
\usepackage{marginnote}
\usepackage{etoolbox}
\usepackage{subcaption}
\usepackage{bbding}
\usepackage{pifont}
\let\Cross\undefined % conflict because marvosym also defines it
\usepackage{marvosym}
\usepackage{bussproofs}
\usepackage{bussproofs}
% \usepackage{algorithmicx}

\makeatletter
\font\uwavefont=lasyb10 scaled 652

\newcommand\colorwave[1][blue]{\bgroup\markoverwith{\lower3\p@\hbox{\uwavefont\textcolor{#1}{\char58}}}\ULon}
\makeatother

\ifx\paperversion\paperversiondraft
\newcommand\createtodoauthor[2]{%
\def\tmpdefault{emptystring}
\expandafter\newcommand\csname #1\endcsname[2][\tmpdefault]{\def\tmp{##1}\ifthenelse{\equal{\tmp}{\tmpdefault}}
   {\todo[linecolor=#2,backgroundcolor=#2,bordercolor=#2]{\textbf{#1:} ##2}}
   {\ifthenelse{\equal{##2}{}}{\colorwave[#2]{##1}\xspace}{\todo[linecolor=#2,backgroundcolor=#2,bordercolor=#2]{\textbf{#1:} ##2}\colorwave[#2]{##1}}}}
\expandafter\newcommand\csname #1f\endcsname[2][\tmpdefault]{
	\smash{\marginnote{
		\todo[inline,linecolor=#2,backgroundcolor=#2,bordercolor=#2]{\textbf{#1 (Figure):} ##2}}}
   }
}
%
\else
\newcommand\createtodoauthor[2]{%
\expandafter\newcommand\csname #1\endcsname[2][]{##1}%
\expandafter\newcommand\csname #1f\endcsname[2][]{##1}%
}%
\fi

% Broaden margins to make room for todo notes
\makeatletter
\patchcmd{\@addmarginpar}{\ifodd\c@page}{\ifodd\c@page\@tempcnta\m@ne}{}{}
\makeatother
\ifx\paperversion\paperversiondraft
  \makeatletter
  \if@ACM@journal
    \geometry{asymmetric}
    \paperwidth=\dimexpr \paperwidth + 3.5cm\relax
    \oddsidemargin=\dimexpr\oddsidemargin + 0cm\relax
    \evensidemargin=\dimexpr\evensidemargin + 0cm\relax
    \marginparwidth=\dimexpr \marginparwidth + 3cm\relax
    \setlength{\marginparwidth}{4.6cm}
    % This makeatletter box helps to move notes to the right
    \makeatletter
    \long\def\@mn@@@marginnote[#1]#2[#3]{%
      \begingroup
        \ifmmode\mn@strut\let\@tempa\mn@vadjust\else
          \if@inlabel\leavevmode\fi
          \ifhmode\mn@strut\let\@tempa\mn@vadjust\else\let\@tempa\mn@vlap\fi
        \fi
        \@tempa{%
          \vbox to\z@{%
            \vss
            \@mn@margintest
            \if@reversemargin\if@tempswa
                \@tempswafalse
              \else
                \@tempswatrue
            \fi\fi
            %\if@tempswa
              \rlap{%
                \if@mn@verbose
                  \PackageInfo{marginnote}{xpos seems to be \@mn@currxpos}%
                \fi
                \begingroup
                  \ifx\@mn@currxpos\relax\else\ifx\@mn@currxpos\@empty\else
                      \kern-\dimexpr\@mn@currxpos\relax
                  \fi\fi
                  \ifx\@mn@currpage\relax
                    \let\@mn@currpage\@ne
                  \fi
                  \if@twoside\ifodd\@mn@currpage\relax
                      \kern\oddsidemargin
                    \else
                      \kern\evensidemargin
                    \fi
                  \else
                    \kern\oddsidemargin
                  \fi
                  \kern 1in
                \endgroup
                \kern\marginnotetextwidth\kern\marginparsep
                \vbox to\z@{\kern\marginnotevadjust\kern #3
                  \vbox to\z@{%
                    \hsize\marginparwidth
                    \linewidth\hsize
                    \kern-\parskip
                    \marginfont\raggedrightmarginnote\strut\hspace{\z@}%
                    \ignorespaces#2\endgraf
                    \vss}%
                  \vss}%
              }%
          }%
        }%
      \endgroup
    }
    \makeatother
  \else
    \paperwidth=\dimexpr \paperwidth + 6cm\relax
    \oddsidemargin=\dimexpr\oddsidemargin + 3cm\relax
    \evensidemargin=\dimexpr\evensidemargin + 3cm\relax
    \marginparwidth=\dimexpr \marginparwidth + 3cm\relax
    \setlength{\marginparwidth}{4.6cm}
  \fi
  \makeatother
\fi

% We use the following color scheme
%
% This scheme is both print-friendly and colorblind safe for
% up to four colors (including the red tones makes it not
% colorblind safe any more)
%
% https://colorbrewer2.org/#type=qualitative&scheme=Paired&n=4

\definecolor{pairedNegOneLightGray}{HTML}{cacaca}
\definecolor{pairedNegTwoDarkGray}{HTML}{827b7b}
\definecolor{pairedOneLightBlue}{HTML}{a6cee3}
\definecolor{pairedTwoDarkBlue}{HTML}{1f78b4}
\definecolor{pairedThreeLightGreen}{HTML}{b2df8a}
\definecolor{pairedFourDarkGreen}{HTML}{33a02c}
\definecolor{pairedFiveLightRed}{HTML}{fb9a99}
\definecolor{pairedSixDarkRed}{HTML}{e31a1c}
\definecolor{yellow}{HTML}{ffcc80}
\definecolor{lavender}{HTML}{e3aaec}

\createtodoauthor{pw}{yellow}
\createtodoauthor{lyx}{lavender}
%%%%%%%%%%%%%%%%%%%%%%%%%%%%%%%%%%%%%%%%%%
% END draft version config
%%%%%%%%%%%%%%%%%%%%%%%%%%%%%%%%%%%%%%%%%%

% Choose the correct font for Unicode
\usepackage{fontspec}

\setmonofont[Scale=0.9]{DejaVu mononoki Symbola Droid}

\usepackage{bbm}
\usepackage{ucs}

\usepackage[capitalize, nameinlink]{cleveref}
\newcommand{\crefrangeconjunction}{--}
\newcommand{\creflastconjunction}{, and~}

%\usepackage[dvipsnames]{xcolor}
\usepackage{tikz-cd}
\usepackage{fancyvrb}
\usepackage{src_tex/agda}

\setlength{\mathindent}{0.4cm} % Left margin of code blocks - also used by some math environments but I hope we don't care
\fvset{xleftmargin=\mathindent}  % insert the left margin in verbatim
\newenvironment{myDisplay}
 {\VerbatimEnvironment
 \begin{verbatim}
 }
 {\end{verbatim}
 }

\newenvironment{ignore}{}{}

\def\tightlist{}  % pandoc output uses this command

% Setup Agda fonts
% https://lists.chalmers.se/pipermail/agda/2016/008662.html
\renewcommand{\AgdaCodeStyle}{\small}
% Use special font families without TeX ligatures for Agda code. (This
% code is inspired by a comment by Enrico Gregorio/egreg:
% https://tex.stackexchange.com/a/103078.) 
\newfontfamily{\AgdaSerifFont}{DejaVu Serif}
\newfontfamily{\AgdaSansSerifFont}{DejaVu Sans}
\newfontfamily{\AgdaMonoFont}{DejaVu Sans Mono}
\newfontfamily{\AgdaTypewriterFont}{DejaVu mononoki Symbola Droid}
\newfontfamily{\DejaVuSerif}{DejaVu Serif}
\newfontfamily{\Cardo}{Cardo}

% Fallbacks for unsupported planes
% https://blog.michael.franzl.name/2014/12/10/xelatex-unicode-font-fallback-unsupported-characters/
% \usepackage{ucharclasses}
% \setTransitionTo{MathematicalAlphanumericSymbols}{\DejaVuSerif}
% \setTransitionTo{MiscellaneousMathematicalSymbolsB}{\Cardo}

\renewcommand{\AgdaSpace}{\texttt{\AgdaMonoFont{} }}
\renewcommand{\AgdaFontStyle}[1]{{\AgdaTypewriterFont{}#1}}
\renewcommand{\AgdaKeywordFontStyle}[1]{{\AgdaMonoFont{}#1}}
\renewcommand{\AgdaStringFontStyle}[1]{{\AgdaMonoFont{}#1}}
\renewcommand{\AgdaCommentFontStyle}[1]{{\AgdaTypewriterFont{}#1}}
\renewcommand{\AgdaBoundFontStyle}[1]{\AgdaTypewriterFont{}#1}

% hyperref should be loaded last
\usepackage{hyperref}
\hypersetup{colorlinks=true, linkcolor=black, citecolor=black, filecolor=black, urlcolor=black}


\newcommand{\blue}[1]{{\color{blue}#1}}
\newcommand{\red}[1]{{\color{red}#1}}
\newcommand{\RubineRed}[1]{{\color{RubineRed}#1}}
\newcommand{\WildStrawberry}[1]{{\color{WildStrawberry}#1}}

%%%%%%%%%%%%%%%%%%%%%%%%%%%%%%%%%%%%%%%%%%%%%%%%%%%%%%%%

\newcommand{\ie}{\textit{i.e.,}}
\newcommand{\eg}{\textit{e.g.,}}

%%%%%%%%%%%%%%%%%%%%%%%%%%%%%%%%%%%%%%%%%%%%%%%%%%%%%%%%

\author[L. Xia]{Li-yao Xia}
\email{ly.xia@ed.ac.uk}
\affiliation{
  \institution{University of Edinburgh}
  \city{Edinburgh}\country{United Kingdom}
}

% Latex seems to expect the author list abbreviation here...
\author[P. Wadler]{Philip Wadler}
\email{wadler@inf.ed.ac.uk}
\affiliation{
  \institution{University of Edinburgh}
  \city{Edinburgh}\country{United Kingdom}
}

\title{Formalizing gradual algebraic effects in Agda}

\begin{document}

\maketitle

TODO: Too many pages. Current page breakdown:

\begin{itemize}
  \item Intro 1 page
  \item Motivation 3 pages
  \item Types and precision 5 pages
  \item Syntax 4 ages:
    \begin{itemize}
    \item Contexts and terms 3 pages
    \item Values 1 page
    \end{itemize}
  \item Operational semantics 11 pages:
    \begin{itemize}
    \item Frames 4 pages
    \item Reduction 5 pages
    \item Progress and evaluation 2 pages
    \end{itemize}
  \item Precision on terms 10 pages
  \item Simulation proof 20 pages
\end{itemize}

\section{Introduction}

Algebraic effects~\citep{plotkin2001semantics} and their handlers are an
approach to computational effects that has seen rapid development in recent
years, inspiring numerous libraries, experimental programming languages,
and even features in industrial programming languages such as WebAssembly, OCaml,
and Haskell.

The problem of typing algebraic effects has been well studied, and solutions
have been implemented in specialized languages with algebraic effects such as
Links, Eff, Koka, and Frank. However, for bigger languages, it remains a
challenge to integrate such \emph{effect systems} with a pre-existing type
system.

Gradual types~\citep{siek2015} provide a lens through which we may study how
the programs using the untyped algebraic effects of today are to interact with
the effect systems of tomorrow. We present a formalization of a
calculus marrying gradual types with algebraic effects.
% (using Core Eff~\citep{bauer2015programming}).

\subsection{Migrating to more statically checked code}

\def\dhandler{\texttt{loghandler}_\texttt{untyped}}
\def\shandler{\texttt{loghandler}_\texttt{typed}}
\def\dclient{\texttt{client}_\texttt{untyped}}
\def\sclient{\texttt{client}_\texttt{typed}}

A key motivation for gradual types is to enable gradual migration
towards more statically typed code.

While Multicore OCaml features effect handlers, but lacks \emph{effect typing}:
its type system does not keep track of effects yet.
For instance, imagine a library providing a logging
handler, and some client for that handler.
The following table shows the types one could give to such functions without
effect typing---like in OCaml, and with effect typing---like in Koka.
The handler expects a computation which uses the \texttt{log} effect, and
produces a pure computation---with the empty \texttt{ε} effect.

If a future version of OCaml is to feature effect typing,
it may well become a challenge to migrate existing code that could
not rely on effect typing. If the friction is too great, this may lead
to a schism, splitting OCaml into OCaml-without-effect-typing and
OCaml-with-effect-typing.

$$
\begin{array}{rl|rl}
  \dhandler & \texttt{: (unit ⇒ unit) ⇒ unit} & \dclient & \texttt{: unit ⇒ unit} \\
  \shandler & \texttt{: (unit ⇒ <log> unit) ⇒ <ε> unit} & \sclient & \texttt{: unit ⇒ <log> unit}
\end{array}
$$

It would thus be desirable to migrate the existing ecosystem progressively.
In our example we might want to migrate the handler first, while keeping the
whole program---with the effect-untyped client---functional. Or vice versa.

$$
\begin{tikzcd}
  & \dhandler\;\dclient & \\
  \shandler\;\dclient & & \dhandler\;\sclient \\
  & \shandler \;\sclient &
  \arrow["\text{migrate handler}"', from=1-2, to=2-1]
  \arrow["\text{migrate client}", from=1-2, to=2-3]
  \arrow["\text{migrate handler}", from=2-3, to=3-2]
  \arrow["\text{migrate client}"', from=2-1, to=3-2]
\end{tikzcd}

$$

When the typed handler is applied to the untyped client,
the composed program is considered well-typed,
and casts are inserted to ensure that the client indeed behaves as expected by
the static argument type of the handler.

\paragraph{Contributions}

\begin{itemize}
  \item We formalize a cast calculus with effect handlers.
    Our work is mechanized in the Agda proof assistant.
    (Syntax (\Cref{types-and-effects,syntax}), operational semantics (\Cref{operational-semantics}).)
    \Cref{types-and-effects}~defines types and the precision relation on types.
  \item We prove that this language satisfies the gradual guarantee
    (\Cref{gradual-guarantee}).
\end{itemize}

\paragraph{TODO}

\begin{itemize}
  \item Is there an interesting claim to make about "proof-relevant" precision?
  \item Should we formalize a surface calculus that compiles to the cast calculus?
  \item Should we prove the second direction of the gradual guarantee? (via determinism?)
\end{itemize}

\paragraph{Gradual guarantee} If $M$ is more precise than $M'$ ($M \le M'$),
and $M$ steps to a value $V$ ($M \to V$), then $M'$ steps to a value $V'$ which
is less precise than $V$ ($M' \to V' \wedge V \le V'$).

\section{Gradual Eff}

\newcommand\ruledef{::=}
\newcommand\rulealt{\;|\;}
\newcommand\tyarr[2]{#1 \to #2}
\newcommand\tyany{\star}
\newcommand\cty[2]{\langle\,#1\,\rangle\,#2}
\newcommand\tynat{\mathrm{nat}}
\newcommand\tybool{\mathrm{bool}}
\newcommand\tyunit{\mathrm{unit}}
\newcommand\effany{\ding{73}}
\newcommand\effop{\mathit{op}}
\newcommand\app[2]{#1\,#2}
\newcommand\lam[2]{\lambda\, #1 .\, #2}
\newcommand\cast[2]{#1 \,\MVAt\, #2}
\newcommand\tbox[2]{#1 \Uparrow #2}
\newcommand\blame{\mathrm{blame}}
\newcommand\perform[2]{#1\left[#2\right]}
\newcommand\handle[2]{\mathbf{with}\,#1\,\mathbf{handle}\,#2}

\begin{figure}
$$
\begin{array}{rrcll}
 \text{Value types} & A, B, C & \ruledef & \tynat \rulealt \tybool \rulealt \tyunit \rulealt \tyarr{A}{P} \rulealt \tyany & \\
 \text{Computation types} & P, Q, R & \ruledef & \cty{E}{A} &  \\
 \text{Effects} & E, F & \ruledef & \effop_1 \dots \effop_n \rulealt \effany & \\
 \text{Ground types} & G, H & \ruledef & \tynat \rulealt \tybool \rulealt \tyunit \rulealt \tyarr{\tyany}{\tyany}
\end{array}
$$
\caption{Types}
\label{fig:types}
\end{figure}

\begin{figure}
$$
\begin{array}{rcll}
 L, M, N & \ruledef & x & \text{Variable} \\
         & \rulealt & \lam{x}{M} & \text{Abstraction} \\
         & \rulealt & \app{L}{M} & \text{Application} \\
         & \rulealt & k & \text{Constant} \\
         & \rulealt & M \oplus N & \text{Operator} \\
         & \rulealt & \cast{M}{p} & \text{Cast} \\
         & \rulealt & \tbox{M}{G} & \text{Box} \\
         & \rulealt & \blame & \text{Blame} \\
         & \rulealt & \perform{\effop}{M} & \text{Operation} \\
         & \rulealt & \handle{\overline{\perform{\effop_i}{x_i} k_i \mapsto N_i}, x \mapsto L}{M} & \text{Handler}
\end{array}
$$
\caption{Term syntax}
\label{fig:term-syntax}
\end{figure}

\begin{figure}
\caption{Term typing}
\label{fig:term-typing}
\end{figure}

\begin{figure}
$$
\begin{array}{rcll}
 p, q, r & \ruledef & \tynat \rulealt \tybool \rulealt \tyunit \rulealt \tyarr{p}{q} \rulealt p \Uparrow G &
\end{array}
$$
\caption{Cast syntax}
\label{fig:cast-syntax}
\end{figure}

\begin{figure}
\begin{prooftree}
    \AxiomC{$\kappa \in \{ \tynat, \tybool, \tyunit \}$}
  \UnaryInfC{$\kappa : \kappa \le \kappa$}
\end{prooftree}
\begin{prooftree}
    \AxiomC{$p : A \le A^\prime$}
    \AxiomC{$q : B \le B^\prime$}
  \BinaryInfC{$\tyarr{p}{q} : \tyarr{A}{P} \le \tyarr{A^\prime}{P^\prime}$}
\end{prooftree}
\begin{prooftree}
    \AxiomC{$p : A \le G$}
  \UnaryInfC{$p\Uparrow G : A \le \tyany$}
\end{prooftree}
\caption{Cast typing $p : A \le B$}
\label{fig:cast-typing}
\end{figure}

% \input{src_tex/Example}
% \input{src_tex/Type}
% \input{src_tex/Core}
% \input{src_tex/Progress}
% \input{src_tex/Prec}
% \input{src_tex/SimAux}
% \input{src_tex/Sim}

\appendix
\section{Appendix}
\label{sec:appendix}

%\input{src_tex/Utils}

\bibliography{references}

\end{document}
