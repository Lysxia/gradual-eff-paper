Algebraic effects~\citep{plotkin2001semantics} and their handlers are an
approach to computational effects that has seen rapid development in recent
years, inspiring numerous libraries, experimental programming languages,
and even features in industrial programming languages such as WebAssembly, OCaml,
and Haskell.

The problem of typing algebraic effects has been well studied, and solutions
have been implemented in specialized languages with algebraic effects such as
Links, Eff, Koka, and Frank. However, for bigger languages, it remains a
challenge to integrate such \emph{effect systems} with a pre-existing type
system.

Gradual types~\citep{siek2015} provide a lens through which we may study how
the programs using the untyped algebraic effects of today are to interact with
the effect systems of tomorrow. We present a formalization of a
calculus marrying gradual types with algebraic effects.
% (using Core Eff~\citep{bauer2015programming}).

\subsection{Migrating to more statically checked code}

A key motivation for gradual types is to enable gradual migration
towards more statically typed code.

While Multicore OCaml features effect handlers, but lacks \emph{effect typing}:
its type system does not keep track of effects yet.
If a future version of OCaml is to feature effect typing,
it may well become a challenge to migrate existing code that could
not rely on effect typing. If the friction is too great, this may lead
to a schism, splitting OCaml into OCaml-without-effect-typing and
OCaml-with-effect-typing.

For instance, imagine a logging library, implemented as a higher-order function
which handles $\eflog$ effect from its client. Without effect typing, we could
assign the library and the client the types shown at the top of \Cref{fig:app-example}.
With effect typing, we could assign them the types shown at the bottom of that figure,
making explicit the effects that the library consumes and produces.

If the library and the client are large artifacts, it is desirable to
perform this migration from untyped to typed code gradually.
At any point during the migration, the program should remain compilable, and
casts are inserted by the compiler at the interface between the dynamically
typed and statically typed fragments. (\Cref{fig:migration-example})

To ensure that such a migration goes smoothly, the addition of type annotations
should not change the behavior of the program. This property
is called \emph{the gradual guarantee}~\citep{siek2015} or \emph{graduality} (TODO Max New).
As we make the types in the program more precise, some type errors may be
revealed, never hidden, and when the program does run successfully, the output
should not change.
